%%%%%%%%%%%%%%%%%%%%%%%%%%%%%%%%%%%%%%%%%%%%%%%%%%%%%%%%
%
% Change the option between square brackets
% depending on the document you have to write:
%
% proposal    for the initial proposal
% review      for the literature review
% progress    for the progress report
% final       for the final 
% 
%%%%%%%%%%%%%%%%%%%%%%%%%%%%%%%%%%%%%%%%%%%%%%%%%%%%%%%%
\documentclass[final]{cmpreport}
\makeatletter
\input{t1pcr.fd}
\makeatother
\setlength{\footnotesep}{3ex}

% Some package I am using. You may not need them
%
\usepackage{rotating}
\usepackage{subfloat}
\usepackage{color}
\usepackage{amsmath}

%\setkeys{Gin}{draft}

%%%%%%%%%%%%%%%%%%%%%%%%%%%%%%%%%%%%%%%%%%%%%%%%%%%%%%%%
%
%  Fill in the fields with:
%
%  your project title
%  your name
%  your registration number
%  your supervisor's name
%
%%%%%%%%%%%%%%%%%%%%%%%%%%%%%%%%%%%%%%%%%%%%%%%%%%%%%%%%
\title{Real-time Ray Tracing}
%%%%%%%%%%%%%%%%%%%%%%%%%%%%%%%%%%%%%%%%%%%%%%%%%%%%%%%%
%
% The author's name is ignored if the following command 
% is not present in the document
%
% Before submitting a PDF of your final report to the 
% project database you may comment out the command
% if you are worried about lack of anonimity.
%
%%%%%%%%%%%%%%%%%%%%%%%%%%%%%%%%%%%%%%%%%%%%%%%%%%%%%%%%
\author{Kiefer Lam}

\registration{100166387}
\supervisor{Dr Stephen Laycock}

%%%%%%%%%%%%%%%%%%%%%%%%%%%%%%%%%%%%%%%%%%%%%%%%%%%%%%%%
%
% Fill in the field with your module code.
% this should be:
%
% for BIS            -> CMP-6012Y
% for BUSINESS STATS -> CMP-6028Y
% for other students -> CMP-6013Y
%
%%%%%%%%%%%%%%%%%%%%%%%%%%%%%%%%%%%%%%%%%%%%%%%%%%%%%%%%
\ccode{CMP-6013Y}
%\ccode{CMP-6012Y / CMP-6013Y / CMP-6028Y}


\summary{
This document explains how to use the class file \texttt{cmpreport.cls} to write your reports.
The class file has been designed to simplify your life; many things are done for you. As a consequence
some commands presented here are specific to the class file whether they are new commands or customized versions
of commonly known \LaTeX\ commands.
}

\acknowledgements{
This section is used to acknowledge whoever's support and contribution.
The command that introduces it is ignored in the project proposal, literature review and progress report. It is used in the
final report,  but  is not compulsory. If you do not
have an acknowledgements command in your preamble then there
won't be any acknowledgement section in the document produced. \emph{Abstract} and \emph{Acknowledgements} sections should fit on the same page. 
}

%%%%%%%%%%%%%%%%%%%%%%%%%%%%%%%%%%%%%%%%%%%%%%%%%%%%%%%%%%%%%%%%%%
%
% If you do not want a list of figures and a list of tables
% to appear after the table of content then uncomment this line 
%
% Note that the class file contains code to avoid
% producing an empty list section (e.g list of figures) if the 
% list is empty (i.e. no figure in document).
%
% The command also prevents inserting a list of figures or tables 
% anywhere else in the document
%
% Some supervisors think that a report should not contain these
% lists. Please ask your supervisor's opinion.
%
%%%%%%%%%%%%%%%%%%%%%%%%%%%%%%%%%%%%%%%%%%%%%%%%%%%%%%%%%%%%%%%%%%
%\nolist

%%%%%%%%%%%%%%%%%%%%%%%%%%%%%%%%%%%%%%%%%%%%%%%%%%%%%%%%%%%%%%%%%%
%
% Comment out if you want your list of figures and list of
% tables on two or more pages, in particular if the lists do not fit 
% on a single page.
%
%%%%%%%%%%%%%%%%%%%%%%%%%%%%%%%%%%%%%%%%%%%%%%%%%%%%%%%%%%%%%%%%%%
\onePageLists

\begin{document}

\section{Introduction}

\subsection{Aims and Motivations}

\subsection{Problems}

\section{Literature Review}

\section{Design and Implementation}

\subsection{OpenCL and OpenGL}
% Write about OpenCL and OpenGL design and implementation architecture here.
% Include interop. Include OpenCL CLang, OpenCL buffers (cl_mem), OpenGL images and buffers, GLSL.
% Include OpenCL Queue and Event architecture
\subsubsection{Host}
% Write about the things that happen on the host side (e.g. creating memory buffers for data, images, etc)
% Main loop, queue and waiting for queue, displaying the results, data transfer

\subsubsection{Device}
% Write about OpenCL kernels, memory management (e.g. memory types: global, constant, private, local),
% Include limitations of OpenCL CLang, CLang datatypes and unique features

\subsection{Kernel Structure}
% Write about the kernels that are used and why there are multiple kernels.
% Include file structure and header files usage, order of kernels and kernel dependencies
% Include why monolithic kernels may be bad

\subsubsection{Ray Trace Kernel}
% Write that the rays are generated here and traced (intersection test with world objects)
% Include that results are saved to a global buffer which will be used in other kernels

\subsubsection{Image Kernel}
% Write that the image is generated in this kernel. The OpenGL image buffer is written to.
% Include that the data is from the ray trace kernel.

\subsubsection{Reset Kernel}
% Write that the results generated in the ray trace kernel are reset here

\subsubsection{Single Independent Kernel}
% Write that due to memory limitations from the results buffer in the ray trace kernel that a single kernel may be necessary
% Include physical memory limit vs cl_mem buffer allocation limit

\subsection{Ray Generation}
% Mention only basic rays will be included in this section

\subsubsection{Primary Rays}
% Talk about eye rays, method of generation. Explain there's a ray for each pixel. Include diagram and pseudocode
% Explain camera movement here (rotation and translation). Include normalisation (pixel coord for ray position and direction)

\subsubsection{Reflection}
% Write about generating reflection rays using the normal of primary ray result. Use pseudocode and diagram.
% Mention secondary rays and recursion

\subsubsection{Refraction}
% Same as reflection

\subsection{Intersection With Objects}
% Write that for an image to be produced, the ray needs to hit an object and return a value
% So explain why we need intersection

\subsubsection{Sphere Intersection}
% Use pseudocode and diagram and sphere struct
% Include which secondary rays are generated as a result

\subsubsection{Triangle Intersection}
% Write about versatility of triangles and how models use them (the reason we use triangles)
% Include that it is relatively slow for complex models with lots of triangles
% Include that there are methods to get around performance issues
% Include diagrams and pseudocode

\subsection{Generating An Image}
% Write how you need to mix light from the different rays to generate the final colour
% Using material information (mention material properties like opacity, reflectivity, diffuse)

\subsubsection{Phong Model}
% Write about base colour from only primary ray

\section{Results and Testing}

\section{Future Work}

\section{Conclusion}

\end{document}